\section{GPU Device}\label{sec:Device Types / GPU Device}

virtio-gpu is a virtio based graphics adapter.  It can operate in 2D
mode and in 3D (virgl) mode.  3D mode will offload rendering ops to
the host gpu and therefore requires a gpu with 3D support on the host
machine.

3D mode is not covered (yet) in this specification, even though it is
mentioned here and there due to some details of the virtual hardware
being designed with 3D mode in mind.

In 2D mode the virtio-gpu device provides support for ARGB Hardware
cursors and multiple scanouts (aka heads).

\subsection{Device ID}\label{sec:Device Types / GPU Device / Device ID}

16

\subsection{Virtqueues}\label{sec:Device Types / GPU Device / Virtqueues}

\begin{description}
\item[0] controlq - queue for sending control commands
\item[1] cursorq - queue for sending cursor updates
\end{description}

Both queues have the same format.  Each request and each response have
a fixed header, followed by command specific data fields.  The
separate cursor queue is the "fast track" for cursor commands
(VIRTIO_GPU_CMD_UPDATE_CURSOR and VIRTIO_GPU_CMD_MOVE_CURSOR), so they
go though without being delayed by time-consuming commands in the
control queue.

\subsection{Feature bits}\label{sec:Device Types / GPU Device / Feature bits}

\begin{description}
\item[VIRTIO_GPU_F_VIRGL (0)] virgl 3D mode is supported.
\item[VIRTIO_GPU_F_EDID  (1)] EDID is supported.
\item[VIRTIO_GPU_F_RESOURCE_UUID (2)] assigning resources UUIDs for export
  to other virtio devices is supported.
\end{description}

\subsection{Device configuration layout}\label{sec:Device Types / GPU Device / Device configuration layout}

GPU device configuration uses the following layout structure and
definitions:

\begin{lstlisting}
#define VIRTIO_GPU_EVENT_DISPLAY (1 << 0)

struct virtio_gpu_config {
        le32 events_read;
        le32 events_clear;
        le32 num_scanouts;
        le32 reserved;
};
\end{lstlisting}

\subsubsection{Device configuration fields}

\begin{description}
\item[\field{events_read}] signals pending events to the driver.  The
  driver MUST NOT write to this field.
\item[\field{events_clear}] clears pending events in the device.
  Writing a '1' into a bit will clear the corresponding bit in
  \field{events_read}, mimicking write-to-clear behavior.
\item[\field{num_scanouts}] specifies the maximum number of scanouts
  supported by the device.  Minimum value is 1, maximum value is 16.
\end{description}

\subsubsection{Events}

\begin{description}
\item[VIRTIO_GPU_EVENT_DISPLAY] Display configuration has changed.
  The driver SHOULD use the VIRTIO_GPU_CMD_GET_DISPLAY_INFO command to
  fetch the information from the device.  In case EDID support is
  negotiated (VIRTIO_GPU_F_EDID feature flag) the device SHOULD also
  fetch the updated EDID blobs using the VIRTIO_GPU_CMD_GET_EDID
  command.
\end{description}

\devicenormative{\subsection}{Device Initialization}{Device Types / GPU Device / Device Initialization}

The driver SHOULD query the display information from the device using
the VIRTIO_GPU_CMD_GET_DISPLAY_INFO command and use that information
for the initial scanout setup.  In case EDID support is negotiated
(VIRTIO_GPU_F_EDID feature flag) the device SHOULD also fetch the EDID
information using the VIRTIO_GPU_CMD_GET_EDID command.  If no
information is available or all displays are disabled the driver MAY
choose to use a fallback, such as 1024x768 at display 0.

\subsection{Device Operation}\label{sec:Device Types / GPU Device / Device Operation}

The virtio-gpu is based around the concept of resources private to the
host, the guest must DMA transfer into these resources. This is a
design requirement in order to interface with future 3D rendering. In
the unaccelerated 2D mode there is no support for DMA transfers from
resources, just to them.

Resources are initially simple 2D resources, consisting of a width,
height and format along with an identifier. The guest must then attach
backing store to the resources in order for DMA transfers to
work. This is like a GART in a real GPU.

\subsubsection{Device Operation: Create a framebuffer and configure scanout}

\begin{itemize*}
\item Create a host resource using VIRTIO_GPU_CMD_RESOURCE_CREATE_2D.
\item Allocate a framebuffer from guest ram, and attach it as backing
  storage to the resource just created, using
  VIRTIO_GPU_CMD_RESOURCE_ATTACH_BACKING.  Scatter lists are
  supported, so the framebuffer doesn't need to be contignous in guest
  physical memory.
\item Use VIRTIO_GPU_CMD_SET_SCANOUT to link the framebuffer to
  a display scanout.
\end{itemize*}

\subsubsection{Device Operation: Update a framebuffer and scanout}

\begin{itemize*}
\item Render to your framebuffer memory.
\item Use VIRTIO_GPU_CMD_TRANSFER_TO_HOST_2D to update the host resource
  from guest memory.
\item Use VIRTIO_GPU_CMD_RESOURCE_FLUSH to flush the updated resource
  to the display.
\end{itemize*}

\subsubsection{Device Operation: Using pageflip}

It is possible to create multiple framebuffers, flip between them
using VIRTIO_GPU_CMD_SET_SCANOUT and VIRTIO_GPU_CMD_RESOURCE_FLUSH,
and update the invisible framebuffer using
VIRTIO_GPU_CMD_TRANSFER_TO_HOST_2D.

\subsubsection{Device Operation: Multihead setup}

In case two or more displays are present there are different ways to
configure things:

\begin{itemize*}
\item Create a single framebuffer, link it to all displays
  (mirroring).
\item Create an framebuffer for each display.
\item Create one big framebuffer, configure scanouts to display a
  different rectangle of that framebuffer each.
\end{itemize*}

\devicenormative{\subsubsection}{Device Operation: Command lifecycle and fencing}{Device Types / GPU Device / Device Operation / Device Operation: Command lifecycle and fencing}

The device MAY process controlq commands asyncronously and return them
to the driver before the processing is complete.  If the driver needs
to know when the processing is finished it can set the
VIRTIO_GPU_FLAG_FENCE flag in the request.  The device MUST finish the
processing before returning the command then.

Note: current qemu implementation does asyncrounous processing only in
3d mode, when offloading the processing to the host gpu.

\subsubsection{Device Operation: Configure mouse cursor}

The mouse cursor image is a normal resource, except that it must be
64x64 in size.  The driver MUST create and populate the resource
(using the usual VIRTIO_GPU_CMD_RESOURCE_CREATE_2D,
VIRTIO_GPU_CMD_RESOURCE_ATTACH_BACKING and
VIRTIO_GPU_CMD_TRANSFER_TO_HOST_2D controlq commands) and make sure they
are completed (using VIRTIO_GPU_FLAG_FENCE).

Then VIRTIO_GPU_CMD_UPDATE_CURSOR can be sent to the cursorq to set
the pointer shape and position.  To move the pointer without updating
the shape use VIRTIO_GPU_CMD_MOVE_CURSOR instead.

\subsubsection{Device Operation: Request header}\label{sec:Device Types / GPU Device / Device Operation / Device Operation: Request header}

All requests and responses on the virt queues have a fixed header
using the following layout structure and definitions:

\begin{lstlisting}
enum virtio_gpu_ctrl_type {

        /* 2d commands */
        VIRTIO_GPU_CMD_GET_DISPLAY_INFO = 0x0100,
        VIRTIO_GPU_CMD_RESOURCE_CREATE_2D,
        VIRTIO_GPU_CMD_RESOURCE_UNREF,
        VIRTIO_GPU_CMD_SET_SCANOUT,
        VIRTIO_GPU_CMD_RESOURCE_FLUSH,
        VIRTIO_GPU_CMD_TRANSFER_TO_HOST_2D,
        VIRTIO_GPU_CMD_RESOURCE_ATTACH_BACKING,
        VIRTIO_GPU_CMD_RESOURCE_DETACH_BACKING,
        VIRTIO_GPU_CMD_GET_CAPSET_INFO,
        VIRTIO_GPU_CMD_GET_CAPSET,
        VIRTIO_GPU_CMD_GET_EDID,
        VIRTIO_GPU_CMD_RESOURCE_ASSIGN_UUID,

        /* 3d commands (OpenGL) */
        VIRTIO_GPU_CMD_CTX_CREATE = 0x0200,
        VIRTIO_GPU_CMD_CTX_DESTROY,
        VIRTIO_GPU_CMD_CTX_ATTACH_RESOURCE,
        VIRTIO_GPU_CMD_CTX_DETACH_RESOURCE,
        VIRTIO_GPU_CMD_RESOURCE_CREATE_3D,
        VIRTIO_GPU_CMD_TRANSFER_TO_HOST_3D,
        VIRTIO_GPU_CMD_TRANSFER_FROM_HOST_3D,
        VIRTIO_GPU_CMD_SUBMIT_3D,

        /* cursor commands */
        VIRTIO_GPU_CMD_UPDATE_CURSOR = 0x0300,
        VIRTIO_GPU_CMD_MOVE_CURSOR,

        /* success responses */
        VIRTIO_GPU_RESP_OK_NODATA = 0x1100,
        VIRTIO_GPU_RESP_OK_DISPLAY_INFO,
        VIRTIO_GPU_RESP_OK_CAPSET_INFO,
        VIRTIO_GPU_RESP_OK_CAPSET,
        VIRTIO_GPU_RESP_OK_EDID,
        VIRTIO_GPU_RESP_OK_RESOURCE_UUID,

        /* error responses */
        VIRTIO_GPU_RESP_ERR_UNSPEC = 0x1200,
        VIRTIO_GPU_RESP_ERR_OUT_OF_MEMORY,
        VIRTIO_GPU_RESP_ERR_INVALID_SCANOUT_ID,
        VIRTIO_GPU_RESP_ERR_INVALID_RESOURCE_ID,
        VIRTIO_GPU_RESP_ERR_INVALID_CONTEXT_ID,
        VIRTIO_GPU_RESP_ERR_INVALID_PARAMETER,
};

#define VIRTIO_GPU_FLAG_FENCE (1 << 0)

struct virtio_gpu_ctrl_hdr {
        le32 type;
        le32 flags;
        le64 fence_id;
        le32 ctx_id;
        le32 padding;
};
\end{lstlisting}

The fixed header \field{struct virtio_gpu_ctrl_hdr} in each
request includes the following fields:

\begin{description}
\item[\field{type}] specifies the type of the driver request
  (VIRTIO_GPU_CMD_*) or device response (VIRTIO_GPU_RESP_*).
\item[\field{flags}] request / response flags.
\item[\field{fence_id}] If the driver sets the VIRTIO_GPU_FLAG_FENCE
  bit in the request \field{flags} field the device MUST:
  \begin{itemize*}
  \item set VIRTIO_GPU_FLAG_FENCE bit in the response,
  \item copy the content of the \field{fence_id} field from the
    request to the response, and
  \item send the response only after command processing is complete.
  \end{itemize*}
\item[\field{ctx_id}] Rendering context (used in 3D mode only).
\end{description}

On success the device will return VIRTIO_GPU_RESP_OK_NODATA in
case there is no payload.  Otherwise the \field{type} field will
indicate the kind of payload.

On error the device will return one of the
VIRTIO_GPU_RESP_ERR_* error codes.

\subsubsection{Device Operation: controlq}\label{sec:Device Types / GPU Device / Device Operation / Device Operation: controlq}

For any coordinates given 0,0 is top left, larger x moves right,
larger y moves down.

\begin{description}

\item[VIRTIO_GPU_CMD_GET_DISPLAY_INFO] Retrieve the current output
  configuration.  No request data (just bare \field{struct
    virtio_gpu_ctrl_hdr}).  Response type is
  VIRTIO_GPU_RESP_OK_DISPLAY_INFO, response data is \field{struct
    virtio_gpu_resp_display_info}.

\begin{lstlisting}
#define VIRTIO_GPU_MAX_SCANOUTS 16

struct virtio_gpu_rect {
        le32 x;
        le32 y;
        le32 width;
        le32 height;
};

struct virtio_gpu_resp_display_info {
        struct virtio_gpu_ctrl_hdr hdr;
        struct virtio_gpu_display_one {
                struct virtio_gpu_rect r;
                le32 enabled;
                le32 flags;
        } pmodes[VIRTIO_GPU_MAX_SCANOUTS];
};
\end{lstlisting}

The response contains a list of per-scanout information.  The info
contains whether the scanout is enabled and what its preferred
position and size is.

The size (fields \field{width} and \field{height}) is similar to the
native panel resolution in EDID display information, except that in
the virtual machine case the size can change when the host window
representing the guest display is gets resized.

The position (fields \field{x} and \field{y}) describe how the
displays are arranged (i.e. which is -- for example -- the left
display).

The \field{enabled} field is set when the user enabled the display.
It is roughly the same as the connected state of a phyiscal display
connector.

\item[VIRTIO_GPU_CMD_GET_EDID] Retrieve the EDID data for a given
  scanout.  Request data is \field{struct virtio_gpu_get_edid}).
  Response type is VIRTIO_GPU_RESP_OK_EDID, response data is
  \field{struct virtio_gpu_resp_edid}.  Support is optional and
  negotiated using the VIRTIO_GPU_F_EDID feature flag.

\begin{lstlisting}
struct virtio_gpu_get_edid {
        struct virtio_gpu_ctrl_hdr hdr;
        le32 scanout;
        le32 padding;
};

struct virtio_gpu_resp_edid {
        struct virtio_gpu_ctrl_hdr hdr;
        le32 size;
        le32 padding;
        u8 edid[1024];
};
\end{lstlisting}

The response contains the EDID display data blob (as specified by
VESA) for the scanout.

\item[VIRTIO_GPU_CMD_RESOURCE_CREATE_2D] Create a 2D resource on the
  host.  Request data is \field{struct virtio_gpu_resource_create_2d}.
  Response type is VIRTIO_GPU_RESP_OK_NODATA.

\begin{lstlisting}
enum virtio_gpu_formats {
        VIRTIO_GPU_FORMAT_B8G8R8A8_UNORM  = 1,
        VIRTIO_GPU_FORMAT_B8G8R8X8_UNORM  = 2,
        VIRTIO_GPU_FORMAT_A8R8G8B8_UNORM  = 3,
        VIRTIO_GPU_FORMAT_X8R8G8B8_UNORM  = 4,

        VIRTIO_GPU_FORMAT_R8G8B8A8_UNORM  = 67,
        VIRTIO_GPU_FORMAT_X8B8G8R8_UNORM  = 68,

        VIRTIO_GPU_FORMAT_A8B8G8R8_UNORM  = 121,
        VIRTIO_GPU_FORMAT_R8G8B8X8_UNORM  = 134,
};

struct virtio_gpu_resource_create_2d {
        struct virtio_gpu_ctrl_hdr hdr;
        le32 resource_id;
        le32 format;
        le32 width;
        le32 height;
};
\end{lstlisting}

This creates a 2D resource on the host with the specified width,
height and format.  The resource ids are generated by the guest.

\item[VIRTIO_GPU_CMD_RESOURCE_UNREF] Destroy a resource.  Request data
  is \field{struct virtio_gpu_resource_unref}.  Response type is
  VIRTIO_GPU_RESP_OK_NODATA.

\begin{lstlisting}
struct virtio_gpu_resource_unref {
        struct virtio_gpu_ctrl_hdr hdr;
        le32 resource_id;
        le32 padding;
};
\end{lstlisting}

This informs the host that a resource is no longer required by the
guest.

\item[VIRTIO_GPU_CMD_SET_SCANOUT] Set the scanout parameters for a
  single output.  Request data is \field{struct
    virtio_gpu_set_scanout}.  Response type is
  VIRTIO_GPU_RESP_OK_NODATA.

\begin{lstlisting}
struct virtio_gpu_set_scanout {
        struct virtio_gpu_ctrl_hdr hdr;
        struct virtio_gpu_rect r;
        le32 scanout_id;
        le32 resource_id;
};
\end{lstlisting}

This sets the scanout parameters for a single scanout. The resource_id
is the resource to be scanned out from, along with a rectangle.

Scanout rectangles must be completely covered by the underlying
resource.  Overlapping (or identical) scanouts are allowed, typical
use case is screen mirroring.

The driver can use resource_id = 0 to disable a scanout.

\item[VIRTIO_GPU_CMD_RESOURCE_FLUSH] Flush a scanout resource Request
  data is \field{struct virtio_gpu_resource_flush}.  Response type is
  VIRTIO_GPU_RESP_OK_NODATA.

\begin{lstlisting}
struct virtio_gpu_resource_flush {
        struct virtio_gpu_ctrl_hdr hdr;
        struct virtio_gpu_rect r;
        le32 resource_id;
        le32 padding;
};
\end{lstlisting}

This flushes a resource to screen.  It takes a rectangle and a
resource id, and flushes any scanouts the resource is being used on.

\item[VIRTIO_GPU_CMD_TRANSFER_TO_HOST_2D] Transfer from guest memory
  to host resource.  Request data is \field{struct
    virtio_gpu_transfer_to_host_2d}.  Response type is
  VIRTIO_GPU_RESP_OK_NODATA.

\begin{lstlisting}
struct virtio_gpu_transfer_to_host_2d {
        struct virtio_gpu_ctrl_hdr hdr;
        struct virtio_gpu_rect r;
        le64 offset;
        le32 resource_id;
        le32 padding;
};
\end{lstlisting}

This takes a resource id along with an destination offset into the
resource, and a box to transfer to the host backing for the resource.

\item[VIRTIO_GPU_CMD_RESOURCE_ATTACH_BACKING] Assign backing pages to
  a resource.  Request data is \field{struct
    virtio_gpu_resource_attach_backing}, followed by \field{struct
    virtio_gpu_mem_entry} entries.  Response type is
  VIRTIO_GPU_RESP_OK_NODATA.

\begin{lstlisting}
struct virtio_gpu_resource_attach_backing {
        struct virtio_gpu_ctrl_hdr hdr;
        le32 resource_id;
        le32 nr_entries;
};

struct virtio_gpu_mem_entry {
        le64 addr;
        le32 length;
        le32 padding;
};
\end{lstlisting}

This assign an array of guest pages as the backing store for a
resource. These pages are then used for the transfer operations for
that resource from that point on.

\item[VIRTIO_GPU_CMD_RESOURCE_DETACH_BACKING] Detach backing pages
  from a resource.  Request data is \field{struct
    virtio_gpu_resource_detach_backing}.  Response type is
  VIRTIO_GPU_RESP_OK_NODATA.

\begin{lstlisting}
struct virtio_gpu_resource_detach_backing {
        struct virtio_gpu_ctrl_hdr hdr;
        le32 resource_id;
        le32 padding;
};
\end{lstlisting}

This detaches any backing pages from a resource, to be used in case of
guest swapping or object destruction.

\item[VIRTIO_GPU_CMD_RESOURCE_ASSIGN_UUID] Creates an exported object from
  a resource. Request data is \field{struct
    virtio_gpu_resource_assign_uuid}.  Response type is
  VIRTIO_GPU_RESP_OK_RESOURCE_UUID, response data is \field{struct
    virtio_gpu_resp_resource_uuid}. Support is optional and negotiated
    using the VIRTIO_GPU_F_RESOURCE_UUID feature flag.

\begin{lstlisting}
struct virtio_gpu_resource_assign_uuid {
        struct virtio_gpu_ctrl_hdr hdr;
        le32 resource_id;
        le32 padding;
};

struct virtio_gpu_resp_resource_uuid {
        struct virtio_gpu_ctrl_hdr hdr;
        u8 uuid[16];
};
\end{lstlisting}

The response contains a UUID which identifies the exported object created from
the host private resource. Note that if the resource has an attached backing,
modifications made to the host private resource through the exported object by
other devices are not visible in the attached backing until they are transferred
into the backing.

\end{description}

\subsubsection{Device Operation: controlq (3d)}\label{sec:Device Types / GPU Device / Device Operation / Device Operation: controlq (3d)}

These commands are supported by the device if the VIRTIO_GPU_F_VIRGL
feature flag is set.

\begin{description}

\item[VIRTIO_GPU_CMD_CTX_CREATE]
\item[VIRTIO_GPU_CMD_CTX_DESTROY]
\item[VIRTIO_GPU_CMD_CTX_ATTACH_RESOURCE]
\item[VIRTIO_GPU_CMD_CTX_DETACH_RESOURCE]
  Manage OpenGL contexts.

\item[VIRTIO_GPU_CMD_RESOURCE_CREATE_3D]
  Create OpenGL resources.

\item[VIRTIO_GPU_CMD_TRANSFER_TO_HOST_3D]
\item[VIRTIO_GPU_CMD_TRANSFER_FROM_HOST_3D]
  Transfer data from and to OpenGL resources.

\item[VIRTIO_GPU_CMD_SUBMIT_3D]
  Submit rendering commands (mesa gallium command stream).

\end{description}

\subsubsection{Device Operation: cursorq}\label{sec:Device Types / GPU Device / Device Operation / Device Operation: cursorq}

Both cursorq commands use the same command struct.

\begin{lstlisting}
struct virtio_gpu_cursor_pos {
        le32 scanout_id;
        le32 x;
        le32 y;
        le32 padding;
};

struct virtio_gpu_update_cursor {
        struct virtio_gpu_ctrl_hdr hdr;
        struct virtio_gpu_cursor_pos pos;
        le32 resource_id;
        le32 hot_x;
        le32 hot_y;
        le32 padding;
};
\end{lstlisting}

\begin{description}

\item[VIRTIO_GPU_CMD_UPDATE_CURSOR]
Update cursor.
Request data is \field{struct virtio_gpu_update_cursor}.
Response type is VIRTIO_GPU_RESP_OK_NODATA.

Full cursor update.  Cursor will be loaded from the specified
\field{resource_id} and will be moved to \field{pos}.  The driver must
transfer the cursor into the resource beforehand (using control queue
commands) and make sure the commands to fill the resource are actually
processed (using fencing).

\item[VIRTIO_GPU_CMD_MOVE_CURSOR]
Move cursor.
Request data is \field{struct virtio_gpu_update_cursor}.
Response type is VIRTIO_GPU_RESP_OK_NODATA.

Move cursor to the place specified in \field{pos}.  The other fields
are not used and will be ignored by the device.

\end{description}

\subsection{VGA Compatibility}\label{sec:Device Types / GPU Device / VGA Compatibility}

Applies to Virtio Over PCI only.  The GPU device can come with and
without VGA compatibility.  The PCI class should be DISPLAY_VGA if VGA
compatibility is present and DISPLAY_OTHER otherwise.

VGA compatibility: PCI region 0 has the linear framebuffer, standard
vga registers are present.  Configuring a scanout
(VIRTIO_GPU_CMD_SET_SCANOUT) switches the device from vga
compatibility mode into native virtio mode.  A reset switches it back
into vga compatibility mode.

Note: qemu implementation also provides bochs dispi interface io ports
and mmio bar at pci region 1 and is therefore fully compatible with
the qemu stdvga (see \href{https://git.qemu-project.org/?p=qemu.git;a=blob;f=docs/specs/standard-vga.txt;hb=HEAD}{docs/specs/standard-vga.txt} in the qemu source tree).
