\section{File System Device}\label{sec:Device Types / File System Device}

The virtio file system device provides file system access.  The device either
directly manages a file system or it acts as a gateway to a remote file system.
The details of how the device implementation accesses files are hidden by the
device interface, allowing for a range of use cases.

Unlike block-level storage devices such as virtio block and SCSI, the virtio
file system device provides file-level access to data.  The device interface is
based on the Linux Filesystem in Userspace (FUSE) protocol.  This consists of
requests for file system traversal and access the files and directories within
it.  The protocol details are defined by \hyperref[intro:FUSE]{FUSE}.

The device acts as the FUSE file system daemon and the driver acts as the FUSE
client mounting the file system.  The virtio file system device provides the
mechanism for transporting FUSE requests, much like /dev/fuse in a traditional
FUSE application.

This section relies on definitions from \hyperref[intro:FUSE]{FUSE}.

\subsection{Device ID}\label{sec:Device Types / File System Device / Device ID}
  26

\subsection{Virtqueues}\label{sec:Device Types / File System Device / Virtqueues}

\begin{description}
\item[0] hiprio
\item[1\ldots n] request queues
\end{description}

\subsection{Feature bits}\label{sec:Device Types / File System Device / Feature bits}

There are currently no feature bits defined.

\subsection{Device configuration layout}\label{sec:Device Types / File System Device / Device configuration layout}

All fields of this configuration are always available.

\begin{lstlisting}
struct virtio_fs_config {
        char tag[36];
        le32 num_request_queues;
};
\end{lstlisting}

\begin{description}
\item[\field{tag}] is the name associated with this file system.  The tag is
    encoded in UTF-8 and padded with NUL bytes if shorter than the
    available space.  This field is not NUL-terminated if the encoded bytes
    take up the entire field.
\item[\field{num_request_queues}] is the total number of request virtqueues
    exposed by the device.  Each virtqueue offers identical functionality and
    there are no ordering guarantees between requests made available on
    different queues.  Use of multiple queues is intended to increase
    performance.
\end{description}

\drivernormative{\subsubsection}{Device configuration layout}{Device Types / File System Device / Device configuration layout}

The driver MUST NOT write to device configuration fields.

The driver MAY use from one up to \field{num_request_queues} request virtqueues.

\devicenormative{\subsubsection}{Device configuration layout}{Device Types / File System Device / Device configuration layout}

The device MUST set \field{num_request_queues} to 1 or greater.

\subsection{Device Initialization}\label{Device Types / File System Device / Device Initialization}

On initialization the driver first discovers the device's virtqueues.  The FUSE
session is started by sending a FUSE\_INIT request as defined by the FUSE
protocol on one request virtqueue.  All virtqueues provide access to the same
FUSE session and therefore only one FUSE\_INIT request is required regardless
of the number of available virtqueues.

\subsection{Device Operation}\label{sec:Device Types / File System Device / Device Operation}

Device operation consists of operating the virtqueues to facilitate file system
access.

The FUSE request types are as follows:
\begin{itemize}
\item Normal requests are made available by the driver on request queues and
      are used by the device.
\item High priority requests (FUSE\_INTERRUPT, FUSE\_FORGET, and
      FUSE\_BATCH\_FORGET) are made available by the driver on the hiprio queue
      so the device is able to process them even if the request queues are
      full.
\end{itemize}

Note that FUSE notification requests are not supported.

\subsubsection{Device Operation: Request Queues}\label{sec:Device Types / File System Device / Device Operation / Device Operation: Request Queues}

The driver enqueues normal requests on an arbitrary request queue. High
priority requests are not placed on request queues.  The device processes
requests in any order.  The driver is responsible for ensuring that ordering
constraints are met by making available a dependent request only after its
prerequisite request has been used.

Requests have the following format with endianness chosen by the driver in the
FUSE\_INIT request used to initiate the session as detailed below:

\begin{lstlisting}
struct virtio_fs_req {
        // Device-readable part
        struct fuse_in_header in;
        u8 datain[];

        // Device-writable part
        struct fuse_out_header out;
        u8 dataout[];
};
\end{lstlisting}

Note that the words "in" and "out" follow the FUSE meaning and do not indicate
the direction of data transfer under VIRTIO.  "In" means input to a request and
"out" means output from processing a request.

\field{in} is the common header for all types of FUSE requests.

\field{datain} consists of request-specific data, if any.  This is identical to
the data read from the /dev/fuse device by a FUSE daemon.

\field{out} is the completion header common to all types of FUSE requests.

\field{dataout} consists of request-specific data, if any.  This is identical
to the data written to the /dev/fuse device by a FUSE daemon.

For example, the full layout of a FUSE\_READ request is as follows:

\begin{lstlisting}
struct virtio_fs_read_req {
        // Device-readable part
        struct fuse_in_header in;
        union {
                struct fuse_read_in readin;
                u8 datain[sizeof(struct fuse_read_in)];
        };

        // Device-writable part
        struct fuse_out_header out;
        u8 dataout[out.len - sizeof(struct fuse_out_header)];
};
\end{lstlisting}

The FUSE protocol documented in \hyperref[intro:FUSE]{FUSE} specifies the set
of request types and their contents.

The endianness of the FUSE protocol session is detectable by inspecting the
uint32\_t \field{in.opcode} field of the FUSE\_INIT request sent by the driver
to the device.  This allows the device to determine whether the session is
little-endian or big-endian.  The next FUSE\_INIT message terminates the
current session and starts a new session with the possibility of changing
endianness.

\subsubsection{Device Operation: High Priority Queue}\label{sec:Device Types / File System Device / Device Operation / Device Operation: High Priority Queue}

The hiprio queue follows the same request format as the request queues.  This
queue only contains FUSE\_INTERRUPT, FUSE\_FORGET, and FUSE\_BATCH\_FORGET
requests.

Interrupt and forget requests have a higher priority than normal requests.  The
separate hiprio queue is used for these requests to ensure they can be
delivered even when all request queues are full.

\devicenormative{\paragraph}{Device Operation: High Priority Queue}{Device Types / File System Device / Device Operation / Device Operation: High Priority Queue}

The device MUST NOT pause processing of the hiprio queue due to activity on a
normal request queue.

The device MAY process request queues concurrently with the hiprio queue.

\drivernormative{\paragraph}{Device Operation: High Priority Queue}{Device Types / File System Device / Device Operation / Device Operation: High Priority Queue}

The driver MUST submit FUSE\_INTERRUPT, FUSE\_FORGET, and FUSE\_BATCH\_FORGET requests solely on the hiprio queue.

The driver MUST not submit normal requests on the hiprio queue.

The driver MUST anticipate that request queues are processed concurrently with the hiprio queue.

\subsubsection{Device Operation: DAX Window}\label{sec:Device Types / File System Device / Device Operation / Device Operation: DAX Window}

FUSE\_READ and FUSE\_WRITE requests transfer file contents between the
driver-provided buffer and the device.  In cases where data transfer is
undesirable, the device can map file contents into the DAX window shared memory
region.  The driver then accesses file contents directly in device-owned memory
without a data transfer.

The DAX Window is an alternative mechanism for accessing file contents.
FUSE\_READ/FUSE\_WRITE requests and DAX Window accesses are possible at the
same time.  Providing the DAX Window is optional for devices.  Using the DAX
Window is optional for drivers.

Shared memory region ID 0 is called the DAX window.  Drivers map this shared
memory region with writeback caching as if it were regular RAM.  The contents
of the DAX window are undefined unless a mapping exists for that range.

The driver maps a file range into the DAX window using the FUSE\_SETUPMAPPING
request.  Alignment constraints for FUSE\_SETUPMAPPING and FUSE\_REMOVEMAPPING
requests are communicated during FUSE\_INIT negotiation.

When a FUSE\_SETUPMAPPING request perfectly overlaps a previous mapping, the
previous mapping is replaced.  When a mapping partially overlaps a previous
mapping, the previous mapping is split into one or two smaller mappings.  When
a mapping is partially unmapped it is also split into one or two smaller
mappings.

Establishing new mappings or splitting existing mappings consumes resources.
If the device runs out of resources the FUSE\_SETUPMAPPING request fails until
resources are available again following FUSE\_REMOVEMAPPING.

After FUSE\_SETUPMAPPING has completed successfully the file range is
accessible from the DAX window at the offset provided by the driver in the
request.  A mapping is removed using the FUSE\_REMOVEMAPPING request.

Data is only guaranteed to be persistent when a FUSE\_FSYNC request is used by
the device after having been made available by the driver following the write.

\devicenormative{\paragraph}{Device Operation: DAX Window}{Device Types / File System Device / Device Operation / Device Operation: DAX Window}

The device MAY provide the DAX Window to memory-mapped access to file contents.  If present, the DAX Window MUST be shared memory region ID 0.

The device MUST support FUSE\_READ and FUSE\_WRITE requests regardless of whether the DAX Window is being used or not.

The device MUST allow mappings that completely or partially overlap existing mappings within the DAX window.

The device MUST reject mappings that would go beyond the end of the DAX window.

\drivernormative{\paragraph}{Device Operation: DAX Window}{Device Types / File System Device / Device Operation / Device Operation: DAX Window}

The driver SHOULD be prepared to find shared memory region ID 0 absent and fall back to FUSE\_READ and FUSE\_WRITE requests.

The driver MAY use both FUSE\_READ/FUSE\_WRITE requests and the DAX Window to access file contents.

The driver MUST NOT access DAX window areas that have not been mapped.

\subsubsection{Security Considerations}\label{sec:Device Types / File System Device / Security Considerations}

The device provides access to a file system containing files owned by one or
more POSIX user ids and group ids.  The device has no secure way of
differentiating between users originating requests via the driver.  Therefore
the device accepts the POSIX user ids and group ids provided by the driver and
security is enforced by the driver rather than the device.  It is nevertheless
possible for devices to implement POSIX user id and group id mapping or
whitelisting to control the ownership and access available to the driver.

File systems containing special files including device nodes and setuid
executable files pose a security concern.  These properties are defined by the
file type and mode, which are set by the driver when creating new files or by
changes at a later time.  These special files present a security risk when the
file system is shared with another machine.  A setuid executable or a device
node placed by a malicious machine make it possible for unprivileged users on
other machines to elevate their privileges through the shared file system.
This issue can be solved on some operating systems using mount options that
ignore special files.  It is also possible for devices to implement
restrictions on special files by refusing their creation.

When the device provides shared access to a file system between multiple
machines, symlink race conditions, exhausting file system capacity, and
overwriting or deleting files used by others are factors to consider.  These
issues have a long history in multi-user operating systems and also apply to
virtio-fs.  They are typically managed at the file system administration level
by providing shared access only to mutually trusted users.

Multiple machines sharing access to a file system are susceptible to timing
side-channel attacks.  By measuring the latency of accesses to file contents or
file system metadata it is possible to infer whether other machines also
accessed the same information.  Short latencies indicate that the information
was cached due to a previous access.  This can reveal sensitive information,
such as whether certain code paths were taken.  The DAX Window provides direct
access to file contents and is therefore a likely target of such attacks.
These attacks are also possible with traditional FUSE requests.  The safest
approach is to avoid sharing file systems between untrusted machines.

\subsubsection{Live migration considerations}\label{sec:Device Types / File System Device / Live Migration Considerations}

When a driver is migrated to a new device it is necessary to consider the FUSE
session and its state.  The continuity of FUSE inode numbers (also known as
nodeids) and fh values is necessary so the driver can continue operation
without disruption.

It is possible to maintain the FUSE session across live migration either by
transferring the state or by redirecting requests from the new device to the
old device where the state resides.  The details of how to achieve this are
implementation-dependent and are not visible at the device interface level.

Maintaining version and feature information negotiated by FUSE\_INIT is
necessary so that no FUSE protocol feature changes are visible to the driver
across live migration.  The FUSE\_INIT information forms part of the FUSE
session state that needs to be transferred during live migration.
